%************************************************
\chapter{Introduction}\label{ch:introduction}
%************************************************

\section{Single-cell RNA-sequencing}

\subsection{The central dogma of biology}

% Central dogma

% Why people want to study single-cells

\subsection{Bulk expression quantification}

% rt-QPCR (one paragraph)

% microarrays (one paragraph)

% RNA-seq (two paragraphs)

\subsection{Single-cell expression quantification methods}

\subsection{Features of scRNA-seq data}

% Mean-variance relationship

% Dropout

\section{Pseudotime \& trajectories}

\subsection{The pseudotime estimation problem}

\subsection{Early applications to bulk expression data}

\subsection{Single-cell trajectories}

\subsection{Existing pseudotime inference algorithms}

Development of many of these methods concurrent with the work of this thesis.


\section{Statistical latent variable models}

\subsection{Principal component analysis}

Principal component analysis (PCA, \cite{jolliffe2002principal}) is a ubiquitous linear dimensionality reduction technique. PCA has several interpretations\footnote{Lior Pachter has an excellent blog post on this at \url{https://liorpachter.wordpress.com/2014/05/26/what-is-principal-component-analysis/}}, but the most common is that of a linear projection to a low-dimensional space such that the variance of the data points in the projected space is maximised. Intuitively, this can be thought of finding a linear subspace of the high-dimensional data space that best ``explains''
the data.

Mathematically, we start with $G$-dimensional data points $\{\mby_n\}, \; n = 1, \ldots, N$ and wish to find the $Q$ principal axes $\mblambda_q, \; q = 1, \ldots, Q$ that act as a mapping from the observed to latent space such that the variance of the input points projected to the latent space is maximal. To find $\mblambda_q$ the eigenvectors of the sample covariance matrix

\begin{equation}
  \mbS = \frac{1}{N}\sum_{n = 1}^N (\mby_n - \bar{\mby}) (\mby_n - \bar{\mby})^T
\end{equation}

are calculated where $\bar{\mby}$ is the sample mean of $\mby_n$. The $Q$ principal axes are then the $Q$ eigenvectors of $\mbS$ with the largest eigenvalues. The latent space projections $\mbz_n$ of each data point may then be obtained via $\mbz_n = \mbLambda (\mby_n - \bar{\mby})$ where $\mbLambda$ is the $Q \times G$ matrix whose rows are $\mblambda_q$ ordered by decreasing eigenvalue.

PCA is frequently used in computational genomics. It is commonly used for visualisation of genomic variation data such as single nucleotide polymorphism (SNP) datasets, where it has an elegant interpretation in terms of the underlying genealogical history of samples \cite{mcvean2009genealogical}.
It is often applied to microarray gene expression data for tasks such as data visualisation (see e.g. \cite{ringner2008principal}) and for clustering samples \cite{yeung2001principal}.

Since the advent of single-cell RNA-sequencing, PCA has been the go-to dimensionality reduction algorithm for exploratory data analysis\footnote{Though faces stiff competition from t-Stochastic Neighbour Embedding (tSNE)}. Examples include latent space projections to understand cell types and hierarchies in the developing lung \cite{treutlein2014reconstructing} and the transcriptional states defining differentiation of embryonic stem cells under different serums \cite{kolodziejczyk2015single}. PCA is also used for clustering cells, either as a preprocessing step prior to clustering algorithms like $k-$means or as part of likelihood-based clustering \cite{yau2016pcareduce}.

PCA also forms an initial step in many pseudotime algorithms such as in TSCAN \cite{ji2016tscan} and Waterfall \cite{shin2015single}. However, no studies have actually considered that a principal component of the data itself could \emph{be} the pseudotemporal trajectory. Intuitively such an idea is appealing since we would expect the pseudotemporal process to be the dominant source of variation within the data.

\subsection{Probabilistic principal components analysis}

One weakness of standard PCA is the absence of any probabilistic framework or interpretation. In a landmark paper \cite{tipping1999probabilistic} Tipping \& Bishop derived\footnote{By first considering a factor analysis model that we discuss in section \ref{sec:intr:fa}.} probabilistic PCA (PPCA) that provides an explicit generative model for PCA and relates the maximum likelihood estimates of parameters to the algorithmic estimation we discussed previously.

The generative model for PPCA is given by

\begin{equation}
  \begin{aligned}
    \mbz_n & \sim \Norm(\mbzero, \mbI) \\
    \mby_n & \sim \Norm(\mbLambda \mbz_n + \mbmu, \sigma^2 \mbI)
  \end{aligned}
\end{equation}

where $\mbmu$ is the expectation of $\mby$ and $\sigma^2$ is the measurement variance. In other words, if we centre $\mby$ so that $\text{E}[\mby] = 0$ then PPCA corresponds to a Gaussian noise model with a mean given by standard PCA and isotropic covariance. Tipping and Bishop futher derived closed form expressions for the maximum likelihood estimates (MLEs) of $\mbLambda$ and $\sigma^2$ along with the conditional distribution of the latent variables $p(\mbz_n | \mbLambda, \mby_n)$. They demonstrated that in the limit $\sigma^2 \rightarrow 0$ the MLE estimate of $\mbLambda$ is identical (up to arbitrary rotation) to that of standard PCA. Furthermore, the MLE estimate of $\sigma^2$ is given by

\begin{equation}
  \sigma^2_{\text{MLE}} = \frac{1}{G-Q} \sum_{j = Q + 1}^G \zeta_j
\end{equation}

where $\zeta_j$ are the eigenvalues of the sample covariance matrix ordered by decreasing size. In other words $\sigma^2$ is the variance ``lost'' in the projection, averaged over the remaining dimensions.

\subsection{Factor analysis} \label{sec:intr:fa}

Factor analysis (FA) precedes both PCA and PPCA, dating back to Spearman's work on ``general intelligence'' \cite{spearman1904general} (see section \ref{intr:fa_hist}). The overall model is very similar to that of PPCA with the only difference being the measurement (co-)variance diagonal rather than isotropic. This gives a generative factor analysis model of the form

\begin{equation} \label{eq:fa}
  \begin{aligned}
    \mbz_n & \sim \Norm(\mbzero, \mbI) \\
    \mby_n & \sim \Norm(\mbLambda \mbz_n + \mbmu, \mbSigma)
  \end{aligned}
\end{equation}

where $\mbSigma = \text{diag}(\sigma_1^2, \ldots, \sigma_G^2)$.

Maximum likelihood inference of factor analysis models may be performed using iterative procedures such as expectation-maximisation (EM) or Bayesian inference performed through MCMC methods or variational inference.

Factor analysis notably suffers from the \emph{rotation problem}. Given a $Q \times Q$ orthogonal rotation matrix $\mbR$ the likelihood is invariant under simultaneous rotation of both the factor matrix $\mbLambda$ and latent values $\mbz$\footnote{
Given measurement noise $\mbepsilon$ the likelihood is unchanged under $\mby = \mbLambda \mbz + \mbepsilon = \mbLambda \mbR \mbR^T \mbz + \mbepsilon = \mbLambda' \mbz' + \mbepsilon$ where $\mbLambda' = \mbLambda \mbR$ and $\mbz' = \mbR \mbz$ are the loadings and projections in the rotated space.
}. %Several solutions have been proposed to this
Note that if $Q=1$ the rotation problem is essentially a scaling problem (we can divide $\mbLambda$ by $k$ and multiply $\mbz$ by $k$ to achieve the same likelihood) but this is solved by fixing the scales of $\mbLambda$ and $\mbz$ through priors.

\subsubsection{Origins of factor analysis and connection to pseudotime} \label{intr:fa_hist}



\begin{table}
  \centering
\subfloat[School test scores for children across subjects]{\begin{tabular}{|c|ccc|}
\hline
Student & Maths & Physics & Biology\\
\hline
A & 8  & 9 & 7 \\
B & 3 & 1 & 5 \\
C & 6 & 6 & 8 \\
D & 4 & 2 & 7 \\
\hline
\end{tabular}}
\hspace{5pt}
\subfloat[Gene expression values of cells for transcription factors]{\begin{tabular}{|c|ccc|}
\hline
Cell & \emph{MYOD} & \emph{MYF5} & \emph{MYOG} \\
\hline
A & 8  & 9 & 7 \\
B & 3 & 1 & 5 \\
C & 6 & 6 & 8 \\
D & 4 & 2 & 7 \\
\hline
\end{tabular}}
\caption{Spearman noted that children's test scores were correlated across subjects (a). Cells' expression of genes in biological pathways is likewise correlated (b).} \label{tbl:fa}
\end{table}

Factor analysis was first introduced by Spearman in 1904 \cite{spearman1904general} by considering children's test scores in subjects such as that in table \ref{tbl:fa}(a). He noticed that the scores are correlated - for example, if a child has a high score in physics they are more likely to have a high score in maths, and vice versa. Spearman's insight was that rather than the scores being correlated with each other, they were correlated with an underlying (one-dimensional) hidden factor he termed \emph{general intelligence}. Mathematically this can be expressed as

\begin{equation}
\mby_n = \mblambda z_n + \epsilon_n
\end{equation}

where $\mby_n$ is the vector of child $n$'s test scores, $\mblambda$ is a vector of subject specific constants (the ``loadings''), $z_n$ is the ``general intelligence'' and $\epsilon_n$ is any noise not explained by the model\footnote{
Of course the idea of a single ``general intelligence'' goes against common sense. If there truly is a single latent factor it is possible to form the \emph{tetrad equations}. Later studies showed that deviations in the tetrad equations could not be explained by sampling noise alone. For an overview see \url{https://www.stat.cmu.edu/~cshalizi/350/lectures/12/lecture-12.pdf}.
}.

One can see in hindight the parallels to the pseudotime estimation problem: we measure some high-dimensional quantity $\mby_n$ over various subjects and have a hazy and somewhat under-defined one-dimensional generating process $z_n$. In the context of psychology, this process is intelligence: something unmeasurable that we use to make sense of observations in life such as performance in tests. Similarly in biology we have the concept of differentiation trajectories: something equally unmeasurable but which we indirectly observe through molecular measurements such as gene expression quantification.

Given the parallels between the original interpretation of factor analysis and differentiation trajectories we may suspect that it is particularly suited to the pseudotime inference. Indeed, chapters 3-5 are devoted to various modifications of factor analysis to infer such trajectories.

% \subsubsection{Nonlinear factor analysis}

% \subsection{Mixture models}


\subsection{Manifold learning}

While not strictly a class of statistical latent variable model, manifold learning has found much success in single-cell genomics. While methods such as PCA and FA attempt to infer low-dimensional \textbf{linear} subspaces embedded in high-dimensional space, the field of manifold learning extends this to the nonlinear setting. Several manifold learning algorithms applied to single-cell genomics are reviewed below and utilised in chapter 2.

\subsubsection{Laplacian eigenmaps}

Laplacian eigenmaps \cite{Belkin2003} are part of a larger class of \emph{spectral methods} including diffusion maps (section \ref{sec:diffusion_maps}). Starting with the $N \times G$ matrix $\mbY$, laplacian eigenmaps seeks a $N \times Q$-dimensional embedding $\mbZ$ with row vectors $\mbz_n$ for each cell through minimisation of the quantity

\begin{equation}
  \sum_{n,n'} W_{n,n'}\lvert \lvert \mbz_n - \mbz_{n'}\rvert \rvert^2
\end{equation}

subject to the constrain that $\mbz_q^T \mbz_q = 1 \; \forall q$ where $\mbz_q$ are the column vectors of $\mbz$.
 $\mbW$ is an $N \times N$ similarity matrix between the samples (cells) with the intuition that if
$W_{n,n'}$ is large then the distance between $\mbz_n$ and $\mbz_{n'}$ is heavily penalised placing them close together in the reduced space. Conversely, if $W_{n,n'}$ is small then large distances between them does not affect the optimisation problem. Solutions for $\mbZ$ can readily be found by solving an eigenvalue equation. % KC check + reference
Laplacian eigenmaps were used for pseudotime inference in the \texttt{embeddr} package \cite{campbell2015laplacian} using a symmetrised $k$-nearest neighbour graph for $\mbW$.


\subsubsection{Diffusion maps} \label{sec:diffusion_maps}

Diffusion maps are closely related to laplacian eigenmaps and have been successfully applied to single-cell RNA-seq data both in the context of visualisation \cite{haghverdi2015diffusion} and pseudotime inference \cite{haghverdi2016diffusion}. The basic idea is to consider points on the manifold in terms of a diffusion process, with a sample more likely to diffuse to one closer to it than further away. It begins by constructing a transition matrix

\begin{equation}
  P_{n',n} = \frac{1}{Z(\mby_{n'})} \exp\left( - \frac{\lvert \lvert \mby_n - \mby_{n'} \rvert \rvert^2}{2 \sigma^2}\right)
\end{equation}

where $Z(\mby_{n'}) = \sum_{n}  \exp\left( - \frac{\lvert \lvert \mby_n - \mby_{n'} \rvert \rvert^2}{2 \sigma^2}\right)$ and $\sigma^2$ is a characteristic length scale. $P_{n',n}$ can be thought of as the probability of transitioning or \emph{diffusing} from cell $n$ to $n'$. A renormalised transition matrix $\tilde{\mbP}$ can then be definied that takes into account the local density of samples in the space. One can then decompose the diffusion distances into a sum over the eigenvectors of $\tilde{\mbP}$
weighted by eigenvalues,
implying that retaining eigenvectors for the first $k$ ordered eigenvalues captures the major structure of the manifold and are therefore useful for visualisation. Haghverdi et al. \cite{haghverdi2015diffusion} further derive a heuristic for selection of the kernel width $\sigma$ in terms of the effective number of neighbours of each cell.

\subsubsection{Multidimensional scaling}

Multidimensional scaling (MDS) has previously been used for the visualisation of large genomic datasets \cite{tzeng2008multidimensional} and more recently was used as the initial dimensionality reduction step for the pseudotime algorithm \texttt{SCORPIUS} \cite{cannoodt2016scorpius}. It is motivated by the problem of trying to place cities on points on a map if we are given the distances between them. Given an $N \times N$ distance matrix $\mbD$ MDS attempts to minimise the quantity

\begin{equation}
  \texttt{Stress}(\mbZ) = \left(
  \sum_{n \neq n' = 1}^N (d_{nn'} - \lvert \lvert \mbz_n - \mbz_{n'} \rvert \rvert)^2
  \right)^{\frac{1}{2}}
\end{equation}

where $\mbz_n$ is the low dimensional embedding of sample $n$. The intuition is if $n$ and $n'$ are close we minimise the distance between  $\mbz_n$  $\mbz_{n'}$ and if $n$ and $n'$ are far apart we need to maximise the distance between  $\mbz_n$  and $\mbz_{n'}$ so that it is as close to $d_{nn'}$ as possible.

\subsubsection{Locally linear embedding}

Locally linear embedding (LLE) was used successfully by \emph{SLICER} \cite{welch2016slicer} as a nonlinear dimensionality step after variable gene selection. LLE begins by defining an $N \times N$ weight matrix $\mbW$ where $W_{nn'}$ represents how useful sample $n'$ is for reconstructing $n$. An optimal $\mbW$ is found by minimising

\begin{equation}
  \sum_{n} \lvert \lvert \mby_n - \sum_{n'} W_{nn'} \mby_{n'} \rvert \rvert^2
\end{equation}

where $\mbW$ is sparsely constrained so that each point is only reconstructed by its $k$ nearest neighbours and so that the row sums of $\mbW$ are 1. The $Q$ (reduced) dimensional reconstructions $\mbz_n, \; n = 1, \ldots, N$ are then found via minimising

\begin{equation}
  C(\mbZ) = \sum_{n} \lvert \lvert \mbz_n - \sum_{n'} W_{nn'} \mbz_{n'} \rvert \rvert^2
\end{equation}

where $\mbW$ is kept fixed in the second optimisation step. In other words, we seek a low dimensional embedding such that the points have approximately the same relationship to each other in the reduced space as in the full ($G$-dimensional) space. Minimisation of $C(\mbZ)$ can subsequently be performed via a sparse eigenvalue problem.

\subsubsection{t-distributed stochastic neighbour embedding}

t-distributed stochastic neighbour embedding (t-SNE) \cite{maaten2008visualizing} has become incredibly popular for the visualisation of single-cell RNA-seq data and as the initial dimensionality step for several pseudotime algorithms. Similarly to diffusion maps\footnote{
Note that t-SNE differs in defining a different kernel width for each data point.
}, it begins by defining a conditional transition matrix

\begin{equation}
  P_{n'|n} = \frac{1}{Z(\mby_{n'})} \exp\left( - \frac{\lvert \lvert \mby_n - \mby_{n'} \rvert \rvert^2}{2 \sigma_n^2}\right)
\end{equation}

which can be interpreted as the probability under a Gaussian likelihood of $n$ choosing $n'$ as its neighbour. This is then symmetrised to form $P_{n'n} = \frac{1}{2N}(  P_{n'|n} +   P_{n|n'})$.

It then defines similarities in the latent space as

\begin{equation}
  Q_{nn'} = \frac{
  (1 + \lvert \lvert \mbz_n - \mbz_{n'} \rvert \rvert^2)^{-1}
  }{
  \sum_{m \neq m'} (1 + \lvert \lvert \mbz_m - \mbz_{m'} \rvert \rvert^2)^{-1}
  }
\end{equation}

which is equivalent to measuring distances in the latent space with a Student-t distribution with one degree of freedom. Values of $\mbz_n$ are found by minimising the Kullback–Leibler (KL) divergence
%from $\mbP$ to $\mbQ$
$\KL{\mbP}{\mbQ} = \sum_{n \neq n'} P_{nn'} \log \frac{P_{nn'}}{Q_{nn'}}$.

It is hard to overstate how popular t-SNE has been for visualising single-cell gene expression data. Examples include as the initial dimensionality reduction step in \texttt{SCUBA} \cite{marco2014bifurcation} or for visualisation of branch structure of single-cell mass cytometry data \cite{setty2016wishbone}. Criticisms of t-SNE include the required specification of the kernel widths $\sigma_n$ (which can be interpreted in terms of an effective number of nearest neighbours or \emph{perplexity}) and the number of iterations of the (stochastic) gradient descent aglorithm.


\subsection{Gaussian process latent variable models}

Gaussian Process Latent Variable Models (GPLVM) are the subject of chapter 2 but are mentioned here for completeness. Technically GPLVM is a form of probabilistic manifold learning that learns an explicit map from the latent space to the observed space but may also be seen as a form of nonlinear factor analysis. In the factor analysis model of equation \ref{eq:fa} typical estimation proceeds by marginalising over $\mbz_n$ to give a marginal likelihood $\mby_n \sim(\mbmu, \mbLambda \mbLambda^T + \mbSigma)$ followed by direct optimisation. However, Lawrence \cite{lawrence2004gaussian} instead marginalised over the mapping $\mbLambda$ through a prior of the form $p(\mbLambda) = \prod_{q=1}^Q \Norm(\mblambda_q | \mbzero, \alpha^{-1} \mbI)$. This introduces a coupling between different samples $\mby_n$. Let $\mbY$ be the full $N \times G$ data matrix with row vectors $\mby_n$ and column vectors $\mby_g$ and $\mbZ$ the $N \times Q$
matrix of latent values with row vectors $\mbz_n$. The likelihood marginalised over the mapping is then

\begin{equation}
  p(\mbY) = \prod_{g = 1}^G \norm(\mby_g | \mbzero, \alpha^{-1} \mbZ \mbZ^T + \mbSigma)
\end{equation}

where $\mbSigma$ is the diagonal noise covariance matrix as before.

Lawrence's key insight was that the term $\alpha^{-1} \mbZ \mbZ^T$ in the covariance matrix represents similarity between difference samples, since the covariance between samples $n$ and $n'$ is $\alpha^{-1} \mbz_n \cdot \mbz_{n'}$. Therefore, it can be replaced with any positive definite \emph{kernel} $k(\mbz_n, \mbz_{n'})$ representing similarity between $\mbz_n$ and $\mbz_{n'}$. Popular examples include the squared exponential kernel

\begin{equation}
k_{\text{SQE}}(\mbz_n, \mbz_{n'}) = \sigma_f^2 \exp\left(-\frac{1}{2l^2} \lvert \lvert \mbz_n - \mbz_{n'} \rvert\rvert^2\right)
\end{equation}

as used in \cite{campbell2016order} or the Matern family such as the $\text{Matern}_{3/2}$ kernel used in \cite{reid2016pseudotime}:

\begin{equation}
  k_{\text{Matern}_{3/2}}(\mbz_n, \mbz_{n'}) =
  \left(
  1 + \sqrt{3} \lvert \mbz_n - \mbz_{n'} \rvert \right) \exp\left( - \sqrt{3} \lvert \mbz_n - \mbz_{n'} \rvert \right).
\end{equation}

GPLVM has been widely applied to single-cell expression data, including to single-cell qPCR data \cite{buettner2012novel} and single-cell RNA-seq \cite{campbell2016order,macaulay2016single}. Latent embeddings inferred using GPLVM are typically under-constrained leading to a number of studies that introduce ``data-driven priors'' to further constrain the model, such incorporating the t-SNE cost function to preserve local structure in GPLVM \cite{van2009preserving} 
