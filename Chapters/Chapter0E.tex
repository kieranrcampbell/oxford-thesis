
\chapter{Inference for covariate-adjusted latent variable models} \label{app:clvm}

\section{Overview}

For both Gibbs sampling and variational inference we require the conditional densities $p(\theta_i | \theta_{-i}, \bY)$.

Recall we have data in the form of an $N \times G$ matrix $\bY$ and an $N \times P$ matrix $\bX$ for $N$ samples, $G$ features and $P$ covariates. Our model is
\begin{equation}
\begin{aligned}
\alpha_{pg} & \sim \norm(0, \tau_\alpha^{-1}) \\
\lambda_g & \sim \norm(0, \tau_\lambda^{-1}) \\
z_n & \sim \norm(q_n, \tau_q^{-1}) \\
\beta_{pg} & \sim \norm(0, \chi_{pg}^{-1}) \\
\chi_{pg}^{-1} & \sim \text{Gamma}(a_\beta, b_\beta) \\
\tau_{g}^{-1} & \sim \text{Gamma}(a, b) \\
%\mu_{g} & \sim \norm(0, \tau_\mu^{-1}) \\
\epsilon_{ng} & \sim \norm(0, \tau_g^{-1}) \\
y_{ng} & =  \mu_g + \sum_p \alpha_{pg} x_{np} + \left( \lambda_g + \sum_p \beta_{pg} x_{np} \right) z_n + \epsilon_{ng}
\end{aligned} \label{eq:clvm_model}
\end{equation}

where $\tau_\alpha$, $\tau_\lambda$, $a$, $b$, $a_\beta$, $b_\beta$, $\tau_q$ are fixed hyperparameters and $q_n$ encodes prior information about $z_n$ if available but typically $q_n = 0 \; \forall n$ in the uninformative case.



The motivation to reintroduce $\mu_g$ even if the columns of $\bY$ are centred is as follows: if we consider (in a frequentist setting) the factor loadings to be fixed and $Y$, $X$ and $Z$ as random variables, then we have
\begin{equation}
E[Y_{ng}] \propto \lambda_g E[Z] + \sum_p \beta_{pg} E[X_{pg}Z_n]
\end{equation}

So if we wish to place an informative prior on $Z$ (ie $E[Z] \neq 0$) then $E[Y] \neq 0$ and we have to introduce gene-specific intercepts. This also demonstrates why it is advantageous to have the covariates as uncorrelated with the desired ``pseudotimes'' as possible: covariance here takes the marginal expectation of $Y$ away from zero, biasing modelling assumptions.


\section{Gibbs updates} \label{app:clvm_gibbs}


The conditional distributions are given below (where $\theta | \cdot$ can be interpreted as the conditional distribution of variable $\theta$ conditioned on \emph{all} other variables and the data). For simplicity we assume the summation is obvious from the variable (ie $\sum_p \equiv \sum_{p=1}^P$, etc).



\paragraph{Conditional distribution of $\bz$}
\begin{equation}
z_n | \cdot \sim \norm \left( \frac{ \sum_g \tau_g k_{ng} (y_{ng} - \mu_{g} - \sum_p \alpha_{pg}x_{np}) + \tau_q q_n}{\sum_g \tau_g k_{ng}^2 + \tau_q}, [\sum_g \tau_g k_{ng}^2 + \tau_q]^{-1} \right)
\end{equation}

where $k_{ng} = \lambda_g + \sum_p \beta_{pg} x_{np}$.

\paragraph{Conditional distribution of $\alpha_{pg}$}
\begin{equation}
\alpha_{pg} | \cdot \sim \norm \left( \frac{
\tau_g \sum_n (y_{ng} - \tilde{\mu}_{ng}^{\alpha_p})x_{np}
}{\tau_g \sum_n x_{np}^2 + \tau_\alpha},
[\tau_g \sum_n x_{np}^2 + \tau_\alpha]^{-1} \right)
\end{equation}

where
\begin{equation}
\tilde{\mu}_{ng}^{\alpha_p} = \mu_g + t_n(\lambda_g + \sum_{p'} \beta_{p'g} x_{np'}) + \sum_{p' \neq p} \alpha_{p'g} x_{np'}
\end{equation}

in which $\sum_{p' \neq p}$ denotes the summation over 1 to $P$ excluding $p$.

\paragraph{Conditional distribution of $\beta_{pg}$}
\begin{equation}
\beta_{pg} | \cdot \sim \norm \left( \frac{
\tau_g \sum_n (y_{ng} - \tilde{\mu}_{ng}^{\beta_p})x_{np} z_n
}{\tau_g \sum_n z_n^2 x_{np}^2 + \chi_{pg}},
[\tau_g \sum_n z_n^2 x_{np}^2 + \chi_{pg}]^{-1} \right)
\end{equation}

where
\begin{equation}
\tilde{\mu}_{ng}^{\beta_p} = \mu_g + z_n \lambda_g  +
\sum_{p'} \alpha_{p'g} x_{np'} + z_n \sum_{p' \neq p} \beta_{p'g} x_{np'}
\end{equation}

\paragraph{Conditional distribution of $\tau_g$}
\begin{equation}
\tau_g | \cdot \sim \text{Gamma}\left(
a + \frac{N}{2}, b + \sum_n \frac{(y_{ng} - \tilde{\mu}^\tau_{ng})^2}{2}
\right)
\end{equation}

where
\begin{equation}
\tilde{\mu}^\tau_{ng} = \mu_g + \sum_p \alpha_{pg} x_{np} + \left( \lambda_g + \sum_p \beta_{pg} x_{np} \right) \lambda_n.
\end{equation}

\paragraph{Conditional distribution of $\chi_{pg}$}
\begin{equation}
\chi_{pg} | \cdot \sim \text{Gamma}\left( a_\beta + \frac{1}{2}, b_\beta + \frac{\beta_{pg}^2}{2}\right)
\end{equation}

\paragraph{Conditional distribution of $\lambda_g$}
\begin{equation}
\lambda_g | \cdot \sim \norm\left(
\frac{\tau_g \sum_n z_n(y_{ng} - \tilde{\mu}^\lambda_{ng})}{\tau_g \sum_n z_n^2 + \tau_\lambda},
[\tau_g \sum_n t_n^2 + \tau_\lambda]^{-1}
\right)
\end{equation}

where
\begin{equation}
\tilde{\mu}^\lambda_{ng} = \mu_g + \sum_p \alpha_{pg} x_{np} + \left( \sum_p \beta_{pg} x_{np} \right) z_n
\end{equation}

\paragraph{Conditional distribution of $\mu_g$}
\begin{equation}
\mu_g | \cdot \sim \norm\left(
\frac{\tau_g \sum_n(y_{ng} - \nu_{ng})}{N \tau_g + \tau_\mu}, [N \tau_g + \tau_\mu]^{-1}
\right)
\end{equation}

where
\begin{equation}
\nu_{ng} = \sum_p \alpha_{pg} x_{np} + \left( \lambda_g + \sum_p \beta_{pg} x_{np} \right) z_n
\end{equation}



\section{Variational inference} \label{app:clvm_vi}

\subsection{Conditional expectations}


\paragraph{Conditional expectation of $\mbz$}
\begin{equation}
\begin{aligned}
\Ex_{-z_n}[\mu_{z_n}\tau_{z_n}] & = \sum_g \Big[ \frac{a_{\tau_g}}{b_{\tau_g}}
\Big( m_{\lambda_g} + \sum_p m_{\beta_{pg}} x_{np} \Big) \\
& \times \Big( y_{ng} - m_{\mu_g} - \sum_p m_{\alpha_{pg}} x_{np} \Big)
\Big] + \tau_q q_n \\
\Ex_{-z_n} [\tau_{z_n}] & = \sum_g \frac{a_{\tau_g}}{b_{\tau_g}}
 \Big(m_{\lambda_g}^2 + s_{\lambda_g}^2 + 2m_{\lambda_g} \sum_p m_{\beta_{pg}} x_{np} \\
 & + \sum_p (m_{\beta_{pg}}^2 + s_{\beta_{pg}}^2) x_{np}^2 +
 \sum_{p,p': p \neq p'} m_{\beta_{pg}} m_{\beta_{p'g}} x_{np}x_{np'}
\Big) + \tau_q
\end{aligned}
\end{equation}

Where we have used the fact that
\begin{equation}
  \begin{aligned}
\Ex_{-z_n}[(\lambda_g + \sum_p \beta_{pg} x_{np})^2] & =  \Big(m_{\lambda_g}^2 + s_{\lambda_g}^2 + 2m_{\lambda_g} \sum_p m_{\beta_{pg}} x_{np} \\
& +
 \sum_p (m_{\beta_{pg}}^2 + s_{\beta_{pg}}^2) x_{np} +
 \sum_{p,p': p \neq p'} m_{\beta_{pg}} m_{\beta_{p'g}} x_{np}x_{np'}
\Big)
\end{aligned}
\end{equation}

and for several variables that $\Ex_{-z_n}[\theta^2] = \text{Var}_{-z_n}[\theta] + \Ex_{-z_n}[\theta]^2$.

\paragraph{Conditional expectation of $\alpha_{pg}$}
\begin{equation}
\begin{aligned}
\Ex_{-\alpha_{pg}}[\mu_{pg} \tau_{pg}] & =
\frac{a_{\tau_g}}{b_{\tau_g}} \sum_n \Big (
y_{ng} - m_{\mu_g} - m_{z_n}( m_{\lambda_g} + \sum_{p'} m_{\beta_{p'g}} x_{np} ) \\
& - \sum_{p' \neq p} m_{\alpha_{p'g}} x_{np'}
\Big) x_{np} \\
\Ex_{-\alpha_{pg}}[\tau_{\alpha_{pg}}] & =
\frac{a_{\tau_g}}{b_{\tau_g}} \sum_n x_{np}^2 + \tau_\alpha
\end{aligned}
\end{equation}

\paragraph{Conditional expectation of $\beta_{pg}$}
\begin{equation}
\begin{aligned}
\Ex_{-\beta_{pg}}[\mu_{\beta_{pg}} \tau_{\beta_{pg}}] & =
\frac{a_{\tau_g}}{b_{\tau_g}} \sum_n \Big[
y_{ng} - m_{\mu_g} - \frac{m_{z_n}^2 + s_{z_n}^2}{m_{z_n}} m_{\lambda_g} \\
& - \sum_{p'} m_{\alpha_{p'g}} x_{np'}
- \frac{m_{z_n}^2 + s_{z_n}^2}{m_{z_n}} \sum_{p' \neq p} m_{\beta_{p'g}} x_{np}
\Big] m_{z_n} x_{np} \\
\Ex_{-\beta_{pg}} [ \tau_{\beta_{pg}} ] & =
\frac{a_{\tau_g}}{b_{\tau_g}} \sum_n (m_{z_n}^2 + s_{z_n}^2) x_{np}^2 +
\frac{a_{\chi_{pg}}}{b_{\chi_{pg}}}
\end{aligned}
\end{equation}

Where in both cases we have used the fact that $\Ex_{-\beta_{pg}} [z_n^2] = \text{Var}_{-\beta_{pg}}[z_n] + \Ex_{-\beta_{pg}}[z_n]^2$.

\paragraph{Conditional expectation of $\tau_g$}
\begin{equation}
\begin{aligned}
\Ex_{-\tau_g}[a_{\tau_g}] & = a + \frac{N}{2} \\
\Ex_{-\tau_g}[b_{\tau_g}] & = b + \frac{1}{2} \sum_n f_{ng}
\end{aligned}
\end{equation}

Where
\begin{equation}
\begin{aligned}
f_{ng}  & =  \Ex_{-\tau_g} \left[ \left(y_{ng} - \mu_g - \sum_p \alpha_{pg} x_{np} - \left( \lambda_g + \sum_p \beta_{pg} x_{np} \right) z_n
\right)^2 \right] \\
 & =   \Ex_{-\tau_g} \Bigg[ 
\mu_g^2 +  % 1
2 \mu_g \sum_p \alpha_{pg} x_{np} % 2 +
2 \mu_g z_n \lambda_g  \\% 3
& + 2 \mu_g z_n \sum_p \beta_{pg} x_{np} % 4
- 2 y_{ng} \mu_g + % 5
+ (\sum_p \alpha_{pg} x_{np})^2 \\ % 6
& 2 z_n \lambda_g \sum_p \alpha_{pg} x_{np} % 7
+ 2 z_n (\sum_p \alpha_{pg} x_{np})(\sum_p\beta_{pg} x_{np}) \\ % 8
& - 2 y_{ng} \sum_p \alpha_{pg} x_{np}  % 9
+ z_n^2 \lambda_g^2 % 10
+ 2 \lambda_g z_n^2 \sum_p \beta_{pg} x_{np} \\ % 11
& - 2 \lambda_g z_n y_{ng}  % 12
+ z_n^2 (\sum_p \beta_{pg}x_{np})^2  \\ % 13
& - 2 y_{ng} z_n \sum_p \beta_{pg}x_{np} % 14
+ y_{ng}^2 % 15
\Bigg]
\end{aligned}
\end{equation}

For this we require the identities
\begin{equation}
\begin{aligned}
\Ex[\theta^2] & = \text{Var}[\theta] + \Ex[\theta]^2 \\
\Ex[(\sum_p \gamma_{pg} x_{np})^2] & = \sum_p x_{np}^2 \Ex[\gamma_{pg}^2] +
\sum_{p,p': p \neq p'} x_{np} x_{np'} \Ex[\gamma_{pg} \gamma_{p'g}]
\end{aligned}
\end{equation}

This gives
\begin{equation}
\begin{aligned}
f_{ng}  & =  
m_{\mu_g}^2 + s_{\mu_g}^2 + \\ % 1
& + 2 m_{\mu_g} \sum_p m_{\alpha_{pg}} x_{np}  \\ % 2
&  + 2 m_{\mu_g} m_{z_n} m_{\lambda_g} \\ % 3
& + 2 m_{\mu_g} m_{z_n} \sum_p m_{\beta_{pg}} x_{np} \\ % 4
& - 2 y_{ng} m_{\mu_g} \\ % 5
& + \sum_p (m_{\alpha_{pg}}^2 + s_{\alpha_{pg}}^2) x_{np}^2 + \sum_{p,p':p\neq p'} m_{\alpha_{pg}} m_{\alpha_{p'g}} x_{np} x_{np'} \\ % 6
& + 2 m_{z_n} m_{\lambda_g} \sum_p m_{\alpha_{pg}} x_{np} \\ % 7
& + 2 m_{z_n} (\sum_p m_{\alpha_{pg}} x_{np})(\sum_p m_{\beta_{pg}} x_{np}) \\ % 8
& - 2 y_{ng} \sum_p m_{\alpha_{pg}} x_{np} \\ % 9
& +  (m_{z_n}^2 + s_{z_n}^2) (m_{\lambda_g}^2 + s_{\lambda_g}^2) \\ % 10
& + 2 (m_{z_n}^2 + s_{z_n}^2) m_{\lambda_g} \sum_p m_{\beta_{pg}} x_{np} \\ % 11
& -2 m_{\lambda_g} m_{z_n} y_{ng} \\ % 12
& +(m_{z_n}^2 + s_{z_n}^2) \Big[  \sum_p (m_{\beta_{pg}}^2 + s_{\beta_{pg}}^2) x_{np}^2 \\
& + \sum_{p,p':p\neq p'} m_{\beta_{pg}} m_{\beta_{p'g}} x_{np} x_{np'} \Big] \\ % 13
& - 2 m_{z_n} y_{ng} \sum_p m_{\beta_{pg}} x_{np} ) \\ % 14
& + y_{ng}^2  % 15
\end{aligned}
\end{equation}


\paragraph{Conditional expectation of $\chi_{pg}$}
\begin{equation}
\begin{aligned}
\Ex_{-\chi_{pg}}[a_{\chi_{pg}}] & = a_\beta + \frac{1}{2} \\
\Ex_{-\chi_{pg}}[b_{\chi_{pg}}] & = b_\beta + \frac{1}{2}\left(
m_{\beta_{pg}}^2 + s_{\beta_{pg}}^2 \right)
\end{aligned}
\end{equation}

where again we have used the fact that $\Ex_{-\chi_{pg}}[\beta_{pg}^2] =
\text{Var}_{-\chi_{pg}}[\beta_{pg}] +
\Ex_{-\chi_{pg}}[\beta_{pg}]^2$.

\paragraph{Conditional expectation of $\lambda_g$}
\begin{equation}
\begin{aligned}
\Ex_{-\lambda_g}[\mu_{\lambda_g} \tau_{\lambda_g}] & = \frac{a_{\tau_g}}{b_{\tau_g}}
\sum_n m_{z_n} \Bigg(
y_{ng} - m_{\mu_g} \\
&  - \sum_{p'} m_{\alpha_{p'g}} x_{np'} -
\frac{m_{z_n}^2 + s_{z_n}^2}{m_{z_n}} (\sum_{p'} m_{\beta_{p'g}} x_{np'})
\Bigg) \\
\Ex_{-\lambda_g}[\tau_{\lambda_g}] & = \frac{a_{\tau_g}}{b_{\tau_g}} \sum_n(m_{z_n}^2 + s_{z_n}^2) + \tau_\lambda
\end{aligned}
\end{equation}

\paragraph{Conditional expectation of $\mu_g$}
\begin{equation}
\begin{aligned}
\Ex_{-\mu_g}[\mu_{\mu_g} \tau_{\mu_g}] & =  \frac{a_{\tau_g}}{b_{\tau_g} }
\sum_n \Bigg(
y_{ng} - \sum_{p'} m_{\alpha_{p'g}} x_{np'} \\
& - m_{z_n}(
m_{\lambda_g} + \sum_{p'} m_{\beta_{p'g}} x_{np'})
\Bigg) \\
\Ex_{-\mu_g}[\tau_{\mu_g}] & =  \frac{a_{\tau_g}}{b_{\tau_g}} N + \tau_\mu
\end{aligned}
\end{equation}

\subsection{Calculating the ELBO}

To assess convergence of the CAVI algorithm we need to calculate the evidence lower bound (ELBO) at every iteration (or every $i^{th}$ iteration). The ELBO is given by
\begin{equation}
\text{ELBO} = \Ex[\log p(\bY | \Theta)] + \Ex[\log p(\Theta)] - \Ex[\log q(\Theta)]
\end{equation}

where all expectations are taken with respect to the approximating distribution $Q(\cdot)$ and $\Theta$ denotes the full parameter set. Note that we are implicitly conditioning on the data wherever appropriate, so $p(\bY | \Theta) \equiv p(\bY | \Theta, \bX)$. For this we require the result that if $\theta \sim \text{Gamma}(a, b)$ then $\Ex[\log \theta] = \phi(a) - \log b$ where $\phi$ is the digamma function $\phi(x) = \frac{\Gamma'(x)}{\Gamma(x)}$.

\paragraph{Derivation of $\Ex[\log p(\bY | \Theta)]$}

We have
\begin{equation}
\log p(\bY | \Theta) = \sum_n \sum_g \log \norm(y_{ng} | \mu_{ng}, \tau_g^{-1})
\end{equation}

where $\mu_{ng} = \mu_g + \sum_p \alpha_{pg} x_{np} + z_n \left( \lambda_g + \sum_p \beta_{pg} x_{np} \right)$. Then
\begin{equation}
\begin{aligned}
\Ex[ \log p(\bY | \Theta) ] & \propto \sum_g \left[ \frac{N}{2} \Ex[\log \tau_g] -
\frac{\Ex[\tau_g]}{2} \sum_n \Ex[ (y_{ng} - \mu_{ng})^2 ] \right] \\
& = \sum_g \left[
\frac{N}{2} (\phi(a_{\tau_g}) - \log b_{\tau_g}) - \frac{a_{\tau_g}}{2 b_{\tau_g}} \sum_n f_{ng}
\right]
\end{aligned}
\end{equation}

where $f_{ng}$ is defined as above and we have dropped additive terms since we are only concerned by changes in the ELBO.

\paragraph{Derivation of  $\Ex[\log p(\Theta)]$}

We consider $\Ex[\log p (z_n)]$ which generalises to all parameters with Gaussian priors. We have
\begin{equation}
\begin{aligned}
\Ex[\log p (z_n)] &= \Ex\left[ \frac{1}{2} \log \frac{\tau_q}{2 \pi} - \frac{\tau_q}{2} (z_n - q_n)^2 \right] \\
& = \frac{1}{2} \log \frac{\tau_q}{2 \pi} - \frac{\tau_q}{2} \Ex[ z_n^2 - 2 z_n q_n + q_n^2 ] \\
& = \frac{1}{2} \log \frac{\tau_q}{2 \pi} - \frac{\tau_q}{2} \left( m_{z_n}^2 + s_{z_n}^2 - 2 m_{z_n} q_n + q_n^2 \right)
\end{aligned}
\end{equation}

Next consider $\Ex[\log p (\tau_g)]$ which generalises to all parameters with Gamma priors. We have
\begin{equation}
\begin{aligned}
\Ex[\log p(\tau_g)] & \propto \Ex[\log(\tau_g^{a-1} e^{-\tau_g b})] \\
& = (a-1) \Ex[\log \tau_g] - b \Ex[\tau_g] \\
& = (a-1)(\phi(a_{\tau_g}) - \log b_{\tau_g}) - \frac{a_{\tau_g}}{b_{\tau_g}} b
\end{aligned}
\end{equation}

Thus the expression across all parameters up to a constant value is given by
\begin{equation}
\begin{aligned}
\Ex[\log p(\Theta)] & \propto - \frac{\tau_q}{2}\sum_n (m_{z_n}^2 + s_{z_n}^2 - 2 m_{z_n} q_n) \\
& -\frac{\tau_\mu}{2} \sum_g (m_{\mu_g}^2 +  s_{\mu_g}^2)  - \frac{\tau_\lambda}{2} \sum_g (m_{\lambda_g}^2 + s_{\lambda_g}^2) \\
& + \sum_g \left[ (a-1)(\phi(a_{\tau_g}) - \log b_{\tau_g}) - \frac{a_{\tau_g}}{b_{\tau_g}} b \right] \\
& - \sum_p \sum_g \Big[ \frac{\tau_\alpha}{2} (m_{\alpha_{pg}}^2 + s_{\alpha_{pg}}^2) + \frac{a_{\chi_{pg}}}{2 b_{\chi_{pg}}} (m_{\beta_{pg}}^2 + s_{\beta_{pg}}^2) \\
& - (a_\beta - 1) (\phi(a_{\chi_{pg}}) - \log b_{\chi_{pg}} ) - \frac{a_{\chi_{pg}}}{b_{\chi_{pg}}} b_\beta \Big]
\end{aligned}
\end{equation}

\paragraph{Derivation of $\Ex[\log q(\Theta)]$}

We consider $\Ex[\log q_z (z_n)]$ which naturally generalises to all parameters whose approximating distributions are Gaussian. We have
\begin{equation}
\begin{aligned}
\Ex[\log q_z(z_n)] & \propto \Ex\left[ -\frac{1}{2} \log s_{z_n}^2 - \frac{1}{2 s_{z_n}^2} (z_n - m_{z_n})^2\right] \\
& = -\frac{1}{2} \log s_{z_n}^2 - \frac{1}{2 s_{z_n}^2} \Ex[ z_n^2 - 2 z_n m_{z_n} + m_{z_n}^2 ] \\
& = -\frac{1}{2} \log s_{z_n}^2 - \frac{1}{2 s_{z_n}^2} \Ex[ m_{z_n}^2 + s_{z_n}^2 - 2 m_{z_n}^2 + m_{z_n}^2 ] \\
& \propto-\frac{1}{2} \log s_{z_n}^2
\end{aligned}
\end{equation}

Similarly we consider $\Ex[\log q_\tau(\tau_g)]$ which generalises to all parameters whose approximating distribution is Gamma. We have
\begin{equation}
\begin{aligned}
\Ex[\log q_\tau(\tau_g)] & 
= \Ex[a_{\tau_g} \log b_{\tau_g} + (a_{\tau_g} - 1) \log \tau_g - \tau_g b_{\tau_g} -\log \Gamma(a_{\tau_g})] \\
& = a_{\tau_g} \log b_{\tau_g} + (a_{\tau_g} - 1)(\phi(a_{\tau_g}) - \log b_{\tau_g}) - a_{\tau_g} - \log \Gamma(a_{\tau_g})
\end{aligned}
\end{equation}


Summing this across all parameters gives
\begin{equation}
\begin{aligned}
\Ex[\log q(\Theta)] &= -\frac{1}{2} \sum_n s_{z_n}^2 \\
& + \sum_g \Big(- \frac{1}{2} s_{\mu_g}^2 - \frac{1}{2} s_{\lambda_g}^2 +
a_{\tau_g} \log b_{\tau_g} + (a_{\tau_g} - 1)(\phi(a_{\tau_g}) - \log b_{\tau_g})  \\
& - a_{\tau_g} - \log \Gamma(a_{\tau_g}) \Big) \\
& + \sum_g \sum_p \Big(
-\frac{1}{2} s_{\alpha_{pg}}^2 - \frac{1}{2} s_{\beta_{pg}}^2 \\
& + a_{\chi_{pg}} \log b_{\chi_{pg}} + (a_{\chi_{pg}} - 1)(\phi(a_{\chi_{pg}}) - \log b_{\chi_{pg}}) - a_{\chi_{pg}} - \log \Gamma(a_{\chi_{pg}})\Big)
\end{aligned}
\end{equation}


\section{Stan code for CGPLVM}

\begin{lstlisting}[language=Stan]
data {
  int N; // number of samples
  int G; // number of features
  vector [N] Y[G]; // gene expression input
  real x[N]; // covariate measurements
  real z_prior[N]; // priors on z
  real<lower = 0> z_sd;
  
  // CGPLVM kernel parameters (fixed)
  real <lower = 0> delta;
  real <lower = 0> eta;
  real <lower = 0> nu;
  real <lower = 0> gamma;
  real <lower = 0> xi;
}

transformed data {
  vector[N] mu;
  for(i in 1:N) mu[i] = 0;
}

parameters {
  real<lower = 0> sigma2[G]; // variance for each gene
  real<lower = 0> lambda[G]; // interaction dependence
  real z[N]; // pseudotimes
}

model {
  matrix[N, N] Sigma[G];
  
  for(i in 1:(N-1)) {
    for(j in (i+1):N) {
      for(g in 1:G) {
        Sigma[g,i,j] = nu * exp(-gamma * pow(z[i] - z[j], 2)) + 
        xi * exp(-lambda[g] * pow(z[i] * x[i] - z[j] * x[j], 2)) +
        delta * exp(-eta * pow(x[i] - x[j], 2));
        
        Sigma[g,j,i] = Sigma[g,i,j];
      }
    }
  }
  for(i in 1:N) {
    for(g in 1:G) {
      Sigma[g,i,i] = delta + nu + xi + sigma2[g] + 1e-6;
    }
  }
  
  for(i in 1:N) {
    z[i] ~ normal(z_prior[i], z_sd);
  }
  
  for(g in 1:G) {
    lambda[g] ~ chi_square(1.0);
    1 / sigma2[g] ~ gamma(0.1, 0.1);
    Y[g] ~ multi_normal(mu, Sigma[g]);
  }
}
\end{lstlisting}
